\documentclass[hidelinks]{article}
\usepackage[a4paper, total={7in, 10in}]{geometry}
\usepackage[dvipsnames]{xcolor}
\usepackage{amsmath}
\usepackage{tikz}
\usepackage{tkz-euclide}
\usepackage[ruled,vlined,linesnumbered]{algorithm2e}
\usepackage[unicode]{hyperref}
\usepackage[all]{hypcap}
\usepackage{fancyhdr}
\usepackage{amssymb} %数学符号宏包
\usepackage{float}     % 引入float宏包
\usepackage{setspace} %间距
\usepackage{algorithm}
\usepackage{algpseudocode}

\usepackage{listings} % 导入listingsutf8包
\usepackage{xcolor}   % 导入xcolor包以支持代码高亮


\lstset{
  language=C++,
  basicstyle=\ttfamily,
  keywordstyle=\color{blue},
  commentstyle=\color{green},     % 设置注释颜色
  stringstyle=\color{red},
  breaklines=true,
  numbers=left,
  numberstyle=\tiny\color{gray},
  showstringspaces=false,
  extendedchars=false,            % 禁用扩展字符
  inputencoding=utf8,             % 使用 UTF-8 编码
  literate={#}{\#}1,               % 转义 #
  literate= {(}{{(}}1 {)}{{)}}1 {;}{{;}}1 {,}{{,}}1 {。}{{.}}1 {‘}{{'}}1 {’}{{'}}1
}

\title{\textbf{MA215 Probability Homework-5}}
\author{HONGLI YE 12311501}
\date{October $24^{th} 2024$}

\begin{document}

\setstretch{1.2} % 设置1.2倍行间距
\hypersetup{bookmarksnumbered=true,}
\pagecolor{white}
\color{black}
\maketitle


%% For Algotirhm.
\SetAlgoNoLine % 去除线段
\SetAlgoNlRelativeSize{0} % 去除左边的线条
\SetAlgoNoEnd % 去除 `End` 标记
%%------

\begin{Large}
\tableofcontents
\end{Large}%
\pagebreak

\section{Question 1}
Suppose that the cumulative distribution function of the random variable $X$ is given
by:
\[
F(x) = 
\begin{cases}
    1 - e^{-2x^2} & x \geq 0 \\
    0 & x < 0
\end{cases}
\]
Calculate: (a)$EX$ (b)$Var(X)$ (c)$E(e^{X^2})$

\textbf{Answer:}\\
By definition: we know that:
$$ F(X \leq b) = 1 - e^{-2b^2} $$
Take the derivative of it, we could get a probability distribution function of $X$.
\[f(b) = F_X'(b) = 
\begin{cases}
    4be^{-2b^2} & b \geq 0\\
    0 & b < 0
\end{cases} \]
\begin{enumerate}
    \item[a)] $$ EX = \int_{-\infty}^{\infty} xf(x) \, dx = \int_{-\infty}^{0} xf(x) \, dx + \int_{0}^{\infty} 4x^2e^{-2x^2} \, dx = \int_{0}^{\infty} 4x^2e^{-2x^2} \, dx$$
    Let $u = 2x^2$, then:
    $$ EX = \frac{1}{\sqrt{2}}\int_{0}^{\infty} u^{\frac{1}{2}}e^{-u} \, du = \frac{1}{\sqrt{2}} \Gamma(\frac{3}{2}) = \frac{\sqrt{2\pi}}{4}$$
    \item[b)] $$ Var(X) = E(X^2) - (EX)^2$$
    $$ E(X^2) = \int_{-\infty}^{\infty} 4x^3e^{-2x^2} \, dx = \int_{0}^{\infty} 2ue^{-2u} \, du \text{ where } u = x^2$$
    $$ E(X^2) = \frac{1}{2}\int_{0}^{\infty} ue^{-u} \, du = \frac{1}{2}\Gamma(2) = \frac{1}{2}$$
    So:
    $$ Var(X) = \frac{1}{2} - \frac{\pi}{8} = \frac{4-\pi}{8}$$
    \item[c)] $$ E(e^{X^2}) =  \int_{-\infty}^{\infty} e^{x^2}f(x) \, dx =  \int_{0}^{\infty} 4xe^{-x^2} \, dx = \int_{0}^{\infty} 2e^{-x^2} \, d(x^2) = -2(0 - e^0) = 2$$
\end{enumerate}

\section{Question 2}
To be a winner in a certain game, you must be successful in three successive rounds.
The game depends on the value of $U$, a uniform random variable on $(0, 1)$. If $U > 0.15$, then you are successful in round 1; if $U > 0.24$, then you are successful in round 2; and
if $U > 0.45$, then you are successful in round 3.
\begin{enumerate}
    \item Find the probability that you are successful in round 1
    \item Find the conditional probability that you are successful in round 2 given that you were successful in round 1.
    \item Find the conditional probability that you are successful in round 3 given that you were successful in rounds 1 and 2.
    \item Find the probability that you are a winner.
\end{enumerate}

\textbf{Answer:}\\
From the discription we know that $U \sim Uniform(0,1)$
Let $A_1$ be the event that successful in round 1, $A_2$ be the event that successful in round 2, $A_3$ be the event that successful in round 3.\\
By observing the structure, it is obvious that $A_3 \subset A_2 \subset A_1$.
\begin{enumerate}
    \item $$ P(A_1) = P(U > 0.15) = P(0.15 \leq U \leq 1) = \int^1_{0.15}1 \, dx = 1 - 0.15 = 0.85$$
    \item By the definition of conditional probability and independence, we know that:
    $$P(A_2 | A_1) = \frac{P(A_2 \cap A_1)}{P(A_1)} = \frac{P(A_2)}{P(A_1)} = \frac{76}{85}$$
    \item By the definition of conditional probability, we know that:
    $$P(A_3 | (A_2 \cap A_1)) = \frac{P(A_3)}{P(A_2)} = \frac{55}{76}$$
    \item Being a winner means being successful in total 3 games, so the probability equals to:
    $$P(A_1 \cap A_2 \cap A_3) = P(A_3) = 0.55 $$
\end{enumerate}






\section{Question 3}
A student is getting ready to take an important oral examination and is concerned
about the possibility of having an “on” day or an “off” day. He figures that if he has
an on day, then each of his examiners will pass him, independently of each other, with
a probability of 0.7, whereas if he has an off day, this probability will be reduced to 0.4.
Suppose that the student will pass the examination if a majority of the examiners pass
him. Assume that the student is twice as likely to have an off day as he is to have an on
day (that is, off day with probability $\frac{2}{3}$, and on day $\frac{1}{3}$). Suppose that the student requests an examination with 7 examiners.
\begin{enumerate}
    \item  What is the mean and variance for the number of the 7 examiners that pass the
student?
    \item  What is the probability for the student to finally pass the examination?
\end{enumerate}
\textbf{Answer:}\\
From the description of the problem, let $X_i$ be the number of the examiners who pass him, then $X_i = Bin(7, p_i)$, where $i$ represents whether the day is "on" or "off".
\begin{enumerate}
    \item From the properties of a binomial random variable. $EX = np$ and $Var(X) = np(1-p)$
    $$EX = \frac{1}{3}EX_1 + \frac{2}{3}EX_2 = \frac{7\times 0.7 + 2\times 7\times 0.4}{3} = 3.5 $$
    $$Var(X) = E(X^2) - (EX)^2 = E(X^2) - (3.5)^2 = \frac{1}{3}E(X^2_1) + \frac{2}{3}E(X^2_2) - (3.5)^2$$
    By complex calculation, we can get:
    $$Var(X) = 2.59$$

    \item The probability of passing the examination is:
    $$P(X \geq 4) = \frac{1}{3}P(X_1 \geq 4) + \frac{2}{3}P(X_2 \geq 4)$$
    $$P(X_i \geq 4) = P(X_i = 4) - P(X_i = 5) - P(X_i = 6) - P(X_i =7)$$
    Meanwhile:
    $$P(X_i \geq 4) = 1 - P(X_i = 0) - P(X_i = 1) - P(X_i = 2) - P(X_i =3)$$
    We have:
    $$ P(X_i = k) = \binom{7}{k}(p_i)^k(1-p_i)^{7-k}$$
    So:
    $$ P(X_1 \geq 4) = \binom{7}{4}(0.7)^4(0.3)^3 + \binom{7}{5}(0.7)^5(0.3)^2 + \binom{7}{6}(0.7)^6(0.3)^1 + \binom{7}{7}(0.7)^7(0.3)^0 = \frac{7^5\times 13}{2^4\times 5^6} \approx 0.873964$$
    $$ P(X_2 \geq 4) = \binom{7}{4}(0.4)^4(0.6)^3 + \binom{7}{5}(0.4)^5(0.6)^2 + \binom{7}{6}(0.4)^6(0.6)^1 + \binom{7}{7}(0.4)^7(0.6)^0 = \frac{2^4\times 283}{5^6} \approx 0.289792$$
    $$P(X \geq 4) = \frac{1}{3}\times \frac{7^5\times 13}{2^4\times 5^6} + \frac{2}{3}\times \frac{2^4\times 283}{5^6} \approx 48.4516\%$$
\end{enumerate}




\section{Question 4}
Suppose there is a room with three windows, only one of which is open. Two birds in the room try to escape through the windows.
\begin{enumerate}
    \item Assume that the first bird is memoryless, that is, the bird randomly pick one of the three windows and attempts to escape. If it fails, it will again randomly pick one window out of the three. Let $X$ be the number of attempts for the first bird. Find $EX$ and $Var(X)$.
    \item Assume that the second bird does have memory, that is, the bird randomly pick one of the three windows, and if it fails, it will randomly pick one of the remaining windows. Let $Y$ be the number of attempts for the second bird. Find $EY$ and $Var(Y)$.
\end{enumerate}
\textbf{Answer:}
\begin{enumerate}
    \item From the description of the bird, let $X$ be the number of attempts the bird needs to escape, then $X \sim Geo(\frac{1}{3})$. Use the property of geometry random variable.
    $$EX = \frac{1 - \frac{1}{3}}{\frac{1}{3}} = 2. Var(X) = \frac{1 - \frac{1}{3}}{(\frac{1}{3})^2} = 6$$
    \item Without loss of generosity, we assume 1 and 2 windows are closed while the $3^{rd}$ window is open. Then $Y \in \{1,2,3\}$
    $$P(Y = 1) = \frac{1}{3}$$
    $$P(Y = 2) = \frac{2}{3} \times \frac{1}{2} = \frac{1}{3}$$
    $$P(Y = 3) = \frac{2}{3} \times \frac{1}{2} \times 1 = \frac{1}{3}$$
    So:
    $$ EY = \Sigma^3_{i=1}P(Y=i)\times i = 2$$
    $$ E(Y^2) = \Sigma^3_{i=1}P(Y=i)\times i^2 = \frac{1 + 4 + 9}{3} =\frac{14}{3}$$
    $$ Var(Y) = E(Y^2) - (EY)^2 = \frac{2}{3}$$
\end{enumerate}



\section{Question 5}
Evidence concerning the guilt or innocence of a defendant in a criminal investigation can be summarized by the value of an exponential random variable $X$ whose mean $\mu$ depends on whether the defendant is guilty. If innocent, $\mu = 1$; if guilty, $\mu = 2$. The deciding judge will rule the defendant guilty if $X>c$ for some suitably chosen value of $c$.
\begin{enumerate}
    \item  If the judge wants to be 98 percent certain that an innocent man will not be convicted, what should be the value of $c$?
    \item  Using the value of $c$ found in part (1), what is the probability that a guilty defendant will be convicted?
\end{enumerate}

\textbf{Answer:}
\begin{enumerate}
    \item Since the man is innocent, $\mu = 1$, by the property of exponential random variable, $\lambda = \frac{1}{\mu} = 1$. So: $X \sim Exp(1)$.
    $$P(x > c) = P(x \geq c) = \int^\infty_c e^{-x}\, dx = e^{-c}$$
    Since we want 98 percent or more, then we need:
    $$ e^{-c} \geq 1 - 0.98$$
    $$ c \leq -ln(0.02) = ln(50) \approx 3.912$$
    \item If the man is guilty, $\mu = 2$, $\lambda = \frac{1}{\mu} = \frac{1}{2}$
    $$P(X > c) = P(x \geq c) = \int^\infty_c \frac{1}{2}e^{-\frac{1}{2}x}\, dx = e^{-\frac{1}{2}c} = \frac{\sqrt{2}}{10}$$
    
\end{enumerate}






\section{Question 6}
We say $X$ is a Weibull random variable with parameters $\nu \in \mathbb{R}$, $\alpha > 0$, $\beta > 0$ if its cumulative distribution function is given by
\[
F(x) = 
\begin{cases}
    0 & x \leq \nu \\
    1 - e^{-(\frac{x - \nu}{\alpha})^{\beta}} & x > \nu
\end{cases}
\]
\begin{enumerate}
    \item  Show that if $Y = e^{(\frac{x - \nu}{\alpha})^{\beta}}$ then $Y$ is an exponential random variable with parameter $\lambda = 1.$
    \item  Show that if $Y$ is an exponential random variable with parameter $\lambda = 1$, then $\xi = \alpha Y^{1/\beta} + \nu$(Notice that $Y = (\frac{\xi - \nu}{\alpha})^{\beta}$)is also a Weibull random variable.
\end{enumerate}

\textbf{Answer:}
\begin{enumerate}
    \item Since $F(x)$ is a cumulative distribution function, then:
    $$ lim_{x \rightarrow \infty}(e^{(\frac{x - \nu}{\alpha})^{\beta}}) = 0 \text{ and }  \text{is not increasing}$$
    Let $F_Y(t)$ be the cumulative distribution function of $Y = {(\frac{x - \nu}{\alpha})^{\beta}}$
    So:
    \begin{align*}
        F_Y(t) &= P(Y \leq t)\\
               &= P({(\frac{x - \nu}{\alpha})^{\beta}} \leq t)\\
               &= P(x \leq x_0)  \text{ where } {(\frac{x_0 - \nu}{\alpha})^{\beta}} = t \\
               &= F_X(x_0) \\
    \end{align*}
    So:
    \[
    F_Y(t) = 
    \begin{cases}
        0 & t \leq 0 \\
        1 - e^{-t} & t > 0
    \end{cases}
    \]
    After taking the derivative of it.
    \[
    f(t) = 
    \begin{cases}
        0 & t \leq 0 \\
        e^{-t} & t > 0
    \end{cases}
    \]
    So, $$Y \sim Exp(1)$$
    \item Since $Y \sim Exp(1)$, so:
    \[
    F_Y(t) = 
    \begin{cases}
        0 & x \leq 0 \\
        1 - e^{-t} & x > 0
    \end{cases}
    \]
    Now we calculate $F_\xi(x)$:
    \begin{align*}
        F_{\xi}(x) &= P(\xi \leq x) \\
                   &= P(\alpha Y^{\frac{1}{\beta}} + \nu \leq x) \\
                   &= P(Y \leq \frac{x - \nu}{\alpha})^{\beta}) \\
                   &= F_Y((\frac{x - \nu}{\alpha})^{\beta})
    \end{align*}    
    So:
    \[
    F_{\xi}(x) = 
    \begin{cases}
    0 & (\frac{x - \nu}{\alpha})^{\beta} \leq 0 \\
    1 - e^{-(\frac{x - \nu}{\alpha})^{\beta}} & (\frac{x - \nu}{\alpha})^{\beta} > 0
    \end{cases}
    \]
    Simplify it, we will get:
    \[
    F_{\xi}(x) = 
    \begin{cases}
    0 & x \leq \nu \\
    1 - e^{-(\frac{x - \nu}{\alpha})^{\beta}} & x > \nu
    \end{cases}
    \]
    By the definition, it is easy to prove $\xi$ is a Weibull random variable.
\end{enumerate}




\end{document}
